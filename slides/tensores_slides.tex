\documentclass[aspectratio=169]{beamer}
% slides/_preamble_slides.tex
% Tema moderno minimalista
\usetheme{default}
\usecolortheme{default}

\usepackage[spanish]{babel}
\usepackage[utf8]{inputenc}
\usepackage[T1]{fontenc}
\usepackage{lmodern}
\usepackage{graphicx}
\usepackage{ifthen}
\graphicspath{{figuras/}{./figuras/}{./}}

% Colores modernos
\definecolor{primaryColor}{RGB}{0, 102, 204}      % Azul profesional
\definecolor{secondaryColor}{RGB}{255, 87, 51}    % Coral/naranja moderno
\definecolor{accentColor}{RGB}{46, 204, 113}      % Verde menta
\definecolor{darkGray}{RGB}{44, 62, 80}           % Gris oscuro
\definecolor{lightGray}{RGB}{236, 240, 241}       % Gris claro
\definecolor{textColor}{RGB}{33, 37, 41}          % Negro suave

% Configuración del tema
\setbeamercolor{structure}{fg=primaryColor}
\setbeamercolor{palette primary}{bg=primaryColor,fg=white}
\setbeamercolor{palette secondary}{bg=secondaryColor,fg=white}
\setbeamercolor{palette tertiary}{bg=darkGray,fg=white}
\setbeamercolor{palette quaternary}{bg=accentColor,fg=white}

% Título y frames
\setbeamercolor{titlelike}{parent=palette primary}
\setbeamercolor{frametitle}{parent=palette primary}
\setbeamercolor{block title}{bg=primaryColor,fg=white}
\setbeamercolor{block body}{bg=lightGray,fg=textColor}
\setbeamercolor{block title alerted}{bg=secondaryColor,fg=white}
\setbeamercolor{block body alerted}{bg=lightGray,fg=textColor}
\setbeamercolor{block title example}{bg=accentColor,fg=white}
\setbeamercolor{block body example}{bg=lightGray,fg=textColor}

% Items y enumeración
\setbeamercolor{item}{fg=primaryColor}
\setbeamercolor{subitem}{fg=secondaryColor}
\setbeamercolor{subsubitem}{fg=accentColor}

% Estilo de los frames
\setbeamertemplate{frametitle}{
  \vspace{0.5em}
  \insertframetitle
  \vspace{0.3em}
}

% Bloques con esquinas redondeadas
\setbeamertemplate{blocks}[rounded][shadow=false]

% Items modernos
\setbeamertemplate{itemize items}[circle]
\setbeamertemplate{enumerate items}[circle]

% Sin símbolos de navegación
\setbeamertemplate{navigation symbols}{}

% Footline moderno - color único azul oscuro
\setbeamertemplate{footline}{
  \leavevmode%
  \hbox{%
    \begin{beamercolorbox}[wd=\paperwidth,ht=2.5ex,dp=1ex,leftskip=.3cm,rightskip=.3cm]{palette tertiary}%
      \usebeamerfont{author in head/foot}\insertshortauthor
      \hfill
      \usebeamerfont{title in head/foot}\insertshorttitle
      \hfill
      \usebeamerfont{date in head/foot}\insertshortdate{}
      \hspace*{2em}
      \insertframenumber{} / \inserttotalframenumber
    \end{beamercolorbox}}%
  \vskip0pt%
}

% Título con diseño moderno
\setbeamertemplate{title page}{
  \vbox{}
  \vfill
  \begingroup
    \centering
    \begin{beamercolorbox}[sep=8pt,center]{title}
      \usebeamerfont{title}\inserttitle\par%
      \ifx\insertsubtitle\@empty%
      \else%
        \vskip0.25em%
        {\usebeamerfont{subtitle}\usebeamercolor[fg]{subtitle}\insertsubtitle\par}%
      \fi%     
    \end{beamercolorbox}%
    \vskip1em\par
    \begin{beamercolorbox}[sep=8pt,center]{author}
      \usebeamerfont{author}\insertauthor
    \end{beamercolorbox}
    \begin{beamercolorbox}[sep=8pt,center]{institute}
      \usebeamerfont{institute}\insertinstitute
    \end{beamercolorbox}
    \begin{beamercolorbox}[sep=8pt,center]{date}
      \usebeamerfont{date}\insertdate
    \end{beamercolorbox}\vskip0.5em
  \endgroup
  \vfill
}

% Tipografía moderna
\setbeamerfont{title}{size=\Large,series=\bfseries}
\setbeamerfont{subtitle}{size=\large}
\setbeamerfont{frametitle}{size=\large,series=\bfseries}
\setbeamerfont{block title}{size=\normalsize,series=\bfseries}

\hypersetup{colorlinks=true, linkcolor=primaryColor, urlcolor=primaryColor, citecolor=secondaryColor}

% Definir entornos en español
\newtheorem{teorema}{Teorema}
\newtheorem{corolario}{Corolario}
\theoremstyle{definition}
\newtheorem{ejemplo}{Ejemplo}
\newtheorem{ejemplos}{Ejemplos}

\usepackage{amsmath, amssymb, amsthm}
\usepackage{listings}

\lstset{
  language=Python,
  basicstyle=\ttfamily\scriptsize,
  keywordstyle=\color{blue},
  commentstyle=\color{green!60!black},
  stringstyle=\color{red},
  showstringspaces=false,
  breaklines=true,
  frame=single,
  numbers=left,
  numberstyle=\tiny\color{gray},
  backgroundcolor=\color{lightGray}
}

\title{Tensores}
\subtitle{Álgebra Lineal}
\author{Jordan Blancas}
\date{\today}

\begin{document}

\begin{frame}
  \titlepage
\end{frame}

\begin{frame}{Contenido}
  \tableofcontents
\end{frame}

\section{Motivación}

\begin{frame}{¿Qué son los tensores?}
  \begin{block}{Empecemos desde lo conocido}
    Ya sabes trabajar con:
    \begin{itemize}
      \item Números: temperatura = 25°C
      \item Vectores: velocidad = $(3, 4, 5)$ m/s
      \item Matrices: $A = \begin{pmatrix} 1 & 2 \\ 3 & 4 \end{pmatrix}$
    \end{itemize}
  \end{block}
  
  \pause
  
  \begin{block}{Pregunta}
    ¿Y si necesitas algo más complejo? Por ejemplo:
    \begin{itemize}
      \item Un video: matriz que cambia en el tiempo (3D)
      \item Relaciones entre matrices
      \item Propiedades elásticas de materiales (conectan 2 vectores con 2 vectores)
    \end{itemize}
  \end{block}
  
  \pause
  
  \begin{alertblock}{Respuesta: TENSORES}
    Los tensores son la generalización natural de escalares, vectores y matrices.
  \end{alertblock}
\end{frame}

\section{Introducción}

\begin{frame}{La jerarquía: de simple a complejo}
  \begin{block}{Clasificación por ``cuántos índices necesitas''}
    \begin{itemize}
      \item \textbf{Escalar} (orden 0): $a$ - un solo número
      \item \textbf{Vector} (orden 1): $v_i$ - necesitas 1 índice: $v_1, v_2, v_3$
      \item \textbf{Matriz} (orden 2): $A_{ij}$ - necesitas 2 índices: $A_{11}, A_{12}, \ldots$
      \item \textbf{Tensor} (orden $\geq 3$): $T_{ijk\ldots}$ - necesitas 3+ índices
    \end{itemize}
  \end{block}
  
  \pause
  
  \begin{block}{Visualización}
    \begin{itemize}
      \item Escalar: un punto
      \item Vector: una lista (1D)
      \item Matriz: una tabla (2D)
      \item Tensor orden 3: un ``cubo'' de números (3D)
      \item Tensor orden 4: ... (difícil de visualizar!)
    \end{itemize}
  \end{block}
\end{frame}

\begin{frame}{Ejemplos físicos}
  \begin{ejemplos}
    \begin{tabular}{ll}
      \textbf{Escalar (orden 0):} & Temperatura, presión, masa \\
      & \\
      \textbf{Vector (orden 1):} & Velocidad: $(v_x, v_y, v_z)$ \\
      & Fuerza: $(F_x, F_y, F_z)$ \\
      & \\
      \textbf{Matriz (orden 2):} & Tensor de esfuerzos: $\sigma_{ij}$ \\
      & Tensor de deformación: $\varepsilon_{ij}$ \\
      & \\
      \textbf{Tensor (orden 4):} & Tensor de elasticidad: $C_{ijkl}$ \\
      & (relaciona esfuerzo con deformación) \\
    \end{tabular}
  \end{ejemplos}
  
  \pause
  
  \begin{alertblock}{En ingeniería civil}
    El tensor de elasticidad $C_{ijkl}$ conecta esfuerzos con deformaciones:
    $$\sigma_{ij} = C_{ijkl} \varepsilon_{kl}$$
  \end{alertblock}
\end{frame}

\section{Notación de Índices}

\begin{frame}{Simplificando la escritura}
  \begin{block}{Problema}
    Escribir sumas es tedioso:
    $$\vec{v} \cdot \vec{w} = v_1 w_1 + v_2 w_2 + v_3 w_3 = \sum_{i=1}^3 v_i w_i$$
  \end{block}
  
  \pause
  
  \begin{alertblock}{Solución: Notación de Einstein}
    Si un índice aparece \textbf{dos veces}, se entiende que se suma:
    $$v_i w_i \quad \text{significa} \quad \sum_{i=1}^n v_i w_i$$
    ¡No hace falta escribir el símbolo $\sum$!
  \end{alertblock}
  
  \pause
  
  \begin{block}{Terminología}
    \begin{itemize}
      \item \textbf{Índice repetido} (o ``mudo''): se suma sobre él
      \item \textbf{Índice libre}: aparece una sola vez, no se suma
    \end{itemize}
  \end{block}
\end{frame}

\begin{frame}{Ejemplos de notación de Einstein}
  \begin{ejemplos}
    \begin{enumerate}
      \item Producto punto:
      $$v_i w_i = v_1 w_1 + v_2 w_2 + v_3 w_3$$
      (índice $i$ repetido $\rightarrow$ suma)
      
      \pause
      
      \item Producto matriz-vector:
      $$(A\vec{x})_i = A_{ij} x_j = A_{i1}x_1 + A_{i2}x_2 + A_{i3}x_3$$
      ($j$ repetido $\rightarrow$ suma; $i$ libre $\rightarrow$ resultado es vector)
      
      \pause
      
      \item Traza de una matriz:
      $$\text{tr}(A) = A_{ii} = A_{11} + A_{22} + A_{33}$$
      ($i$ repetido en ambos lugares $\rightarrow$ suma)
      
      \pause
      
      \item Norma al cuadrado:
      $$\|\vec{v}\|^2 = v_i v_i = v_1^2 + v_2^2 + v_3^2$$
    \end{enumerate}
  \end{ejemplos}
\end{frame}

\begin{frame}{Reglas importantes}
  \begin{block}{Regla 1: Índice repetido = suma}
    $$a_i b_i \quad \Rightarrow \quad \text{suma sobre } i$$
  \end{block}
  
  \begin{block}{Regla 2: Índice libre = dimensión del resultado}
    $$c_i = A_{ij} b_j$$
    \begin{itemize}
      \item $j$ repetido $\rightarrow$ se suma
      \item $i$ libre $\rightarrow$ resultado es un vector (1 índice libre)
    \end{itemize}
  \end{block}
  
  \pause
  
  \begin{alertblock}{Advertencia}
    Un índice NO puede aparecer más de 2 veces en un mismo término.
    $$A_{ii} B_{ii} \quad \text{$\times$ (incorrecto)}$$
    $$A_{ij} B_{ji} \quad \text{$\checkmark$ (correcto)}$$
  \end{alertblock}
\end{frame}

\section{Producto Tensorial}

\begin{frame}{¿Cómo crear una matriz desde dos vectores?}
  \begin{block}{Idea}
    Dados dos vectores $\vec{u}$ y $\vec{v}$, ¿podemos crear una matriz a partir de ellos?
  \end{block}
  
  \pause
  
  \begin{block}{Producto tensorial: $\vec{u} \otimes \vec{v}$}
    Multiplicar \textbf{cada} componente de $\vec{u}$ con \textbf{cada} componente de $\vec{v}$:
    $$(\vec{u} \otimes \vec{v})_{ij} = u_i v_j$$
  \end{block}
  
  \pause
  
  \begin{ejemplo}
    $$\vec{u} = \begin{pmatrix} 1 \\ 2 \end{pmatrix}, \quad \vec{v} = \begin{pmatrix} 3 \\ 4 \end{pmatrix}$$
    $$\vec{u} \otimes \vec{v} = \begin{pmatrix} 1 \cdot 3 & 1 \cdot 4 \\ 2 \cdot 3 & 2 \cdot 4 \end{pmatrix} = \begin{pmatrix} 3 & 4 \\ 6 & 8 \end{pmatrix}$$
  \end{ejemplo}
\end{frame}

\begin{frame}{Propiedades del producto tensorial}
  \begin{block}{Propiedades importantes}
    \begin{enumerate}
      \item $\text{rango}(\vec{u} \otimes \vec{v}) = 1$ (siempre rango 1)
      \item $(\alpha \vec{u}) \otimes \vec{v} = \alpha (\vec{u} \otimes \vec{v})$
      \item $(\vec{u} + \vec{w}) \otimes \vec{v} = \vec{u} \otimes \vec{v} + \vec{w} \otimes \vec{v}$
      \item NO es conmutativo: $\vec{u} \otimes \vec{v} \neq \vec{v} \otimes \vec{u}$
    \end{enumerate}
  \end{block}
  
  \pause
  
  \begin{alertblock}{Hecho importante}
    Cualquier matriz de rango 1 se puede escribir como producto tensorial de dos vectores.
  \end{alertblock}
  
  \pause
  
  \begin{block}{En notación de índices}
    $$(\vec{u} \otimes \vec{v})_{ij} = u_i v_j$$
    Sin índices repetidos $\rightarrow$ no hay suma $\rightarrow$ resultado es matriz (2 índices)
  \end{block}
\end{frame}

\section{Operaciones con Tensores}

\begin{frame}{¿Qué es la contracción?}
  \begin{block}{Idea}
    La \textbf{contracción} es ``reducir'' un tensor sumando sobre un par de índices.
    
    Es como convertir algo complejo en algo más simple.
  \end{block}
  
  \pause
  
  \begin{ejemplos}[Casos que ya conoces]
    \begin{itemize}
      \item \textbf{Matriz $\times$ vector}: $w_i = A_{ij} v_j$
      
      Tensor orden 2 + vector $\rightarrow$ vector (orden 1)
      
      \pause
      
      \item \textbf{Traza}: $\text{tr}(A) = A_{ii}$
      
      Matriz (orden 2) $\rightarrow$ escalar (orden 0)
      
      \pause
      
      \item \textbf{Producto Frobenius}: $A : B = A_{ij} B_{ij}$
      
      Dos matrices $\rightarrow$ escalar
    \end{itemize}
  \end{ejemplos}
\end{frame}

\begin{frame}{Contracción: Definición general}
  \begin{block}{Definición}
    \textbf{Contraer} = sumar sobre un par de índices:
    $$C_{ik} = T_{ijk} v_j = \sum_{j} T_{ijk} v_j$$
  \end{block}
  
  \pause
  
  \begin{block}{Regla general}
    Si contraes tensor orden $m$ con tensor orden $n$:
    $$\text{Resultado tiene orden } (m + n - 2)$$
    
    Porque pierdes 1 índice de cada uno al sumar.
  \end{block}
  
  \pause
  
  \begin{ejemplos}
    \begin{itemize}
      \item Orden 2 + orden 1: $2 + 1 - 2 = 1$ (matriz $\times$ vector = vector) $\checkmark$
      \item Orden 2 + orden 2: $2 + 2 - 2 = 2$ (matriz $\times$ matriz = matriz) $\checkmark$
      \item Contracción total (traza): $2 + 0 - 2 = 0$ (matriz $\rightarrow$ escalar) $\checkmark$
    \end{itemize}
  \end{ejemplos}
\end{frame}

\section{Tensor de Esfuerzos}

\begin{frame}{Aplicación: Tensor de esfuerzos}
  \begin{block}{Situación}
    En un material, el esfuerzo en diferentes direcciones se describe con una matriz $3 \times 3$:
    $$\sigma_{ij} = \begin{pmatrix}
      \sigma_{xx} & \sigma_{xy} & \sigma_{xz} \\
      \sigma_{yx} & \sigma_{yy} & \sigma_{yz} \\
      \sigma_{zx} & \sigma_{zy} & \sigma_{zz}
    \end{pmatrix}$$
    
    donde $\sigma_{ij}$ = esfuerzo en dirección $j$ sobre plano perpendicular a $i$
  \end{block}
  
  \pause
  
  \begin{block}{Propiedades}
    \begin{itemize}
      \item Simétrico: $\sigma_{ij} = \sigma_{ji}$ (por equilibrio de momentos)
      \item Traza: $\sigma_{ii} = \sigma_{xx} + \sigma_{yy} + \sigma_{zz}$ = presión hidráulica $\times 3$
    \end{itemize}
  \end{block}
\end{frame}

\section{Transformación de Tensores}

\begin{frame}{Cambio de coordenadas}
  \begin{block}{Problema}
    ¿Qué pasa con un tensor cuando rotas el sistema de coordenadas?
  \end{block}
  
  \pause
  
  \begin{block}{Transformación de vector}
    Si rotas coordenadas con matriz $R$:
    $$v'_i = R_{ij} v_j$$
  \end{block}
  
  \pause
  
  \begin{block}{Transformación de tensor orden 2}
    Aplicas la rotación a CADA índice:
    $$T'_{ij} = R_{ik} R_{jl} T_{kl}$$
    
    En notación matricial: $T' = R T R^T$
  \end{block}
  
  \pause
  
  \begin{alertblock}{Invariantes}
    Cantidades que NO cambian con rotación:
    \begin{itemize}
      \item Traza: $T_{ii}$
      \item Determinante: $\det(T)$
      \item Autovalores
    \end{itemize}
  \end{alertblock}
\end{frame}

\section{Código Python}

\begin{frame}[fragile]{Producto Tensorial en NumPy}
\begin{lstlisting}
import numpy as np

# Vectores
u = np.array([1, 2, 3])
v = np.array([4, 5])

# Producto tensorial (outer product)
T = np.outer(u, v)

print("u:", u)
print("v:", v)
print("\nProducto tensorial u ⊗ v:")
print(T)
print("\nForma:", T.shape)  # (3, 2)

# Verificar: T[i,j] = u[i] * v[j]
print("\nVerificacion T[1,0] =", T[1,0], "=", u[1], "*", v[0])

# Rango de la matriz
rango = np.linalg.matrix_rank(T)
print(f"\nRango de la matriz: {rango}")  # Siempre 1
\end{lstlisting}
\end{frame}

\begin{frame}[fragile]{Contracción de Tensores}
\begin{lstlisting}
import numpy as np

# Tensor 3D (2x3x4)
T = np.random.rand(2, 3, 4)

# Vector para contraer
v = np.array([1, 2, 3])

# Contracccion sobre segundo indice: C[i,k] = T[i,j,k] * v[j]
C = np.einsum('ijk,j->ik', T, v)

print("Tensor original:", T.shape)  # (2, 3, 4)
print("Vector:", v.shape)            # (3,)
print("Resultado:", C.shape)         # (2, 4)

# Verificacion manual para primer elemento
manual = sum(T[0, j, 0] * v[j] for j in range(3))
print(f"\nC[0,0] = {C[0,0]:.4f}")
print(f"Verificacion manual: {manual:.4f}")

# Traza de matriz 2D
A = np.array([[1, 2], [3, 4]])
traza = np.einsum('ii', A)  # Suma diagonal
print(f"\nTraza de A: {traza}")  # 5
\end{lstlisting}
\end{frame}

\begin{frame}[fragile]{Transformación de Tensor (Rotación)}
\begin{lstlisting}
import numpy as np

# Tensor simetrico 2x2 (ej: tensor de esfuerzos)
sigma = np.array([[100, 30],
                  [ 30, 50]])

# Matriz de rotacion 45 grados
theta = np.pi/4
R = np.array([[np.cos(theta), -np.sin(theta)],
              [np.sin(theta),  np.cos(theta)]])

# Transformar: sigma' = R @ sigma @ R^T
sigma_rot = R @ sigma @ R.T

print("Tensor original:")
print(sigma)
print("\nMatriz de rotacion:")
print(R)
print("\nTensor rotado:")
print(sigma_rot)

# Verificar invariante (traza)
print(f"\nTraza original: {np.trace(sigma):.2f}")
print(f"Traza rotada: {np.trace(sigma_rot):.2f}")

# Autovalores (tambien invariantes)
print(f"\nAutovalores originales: {np.linalg.eigvals(sigma)}")
print(f"Autovalores rotados: {np.linalg.eigvals(sigma_rot)}")
\end{lstlisting}
\end{frame}

\section{Resumen}

\begin{frame}{Resumen: El camino recorrido}
  \begin{block}{1. La jerarquía}
    Escalar (0) $\rightarrow$ Vector (1) $\rightarrow$ Matriz (2) $\rightarrow$ Tensor (3+)
    
    Número de índices = orden del tensor
  \end{block}
  
  \pause
  
  \begin{block}{2. Notación de Einstein}
    Índice repetido = suma automática: $v_i w_i = \sum_i v_i w_i$
  \end{block}
  
  \pause
  
  \begin{block}{3. Producto tensorial}
    Crear tensor de orden mayor: $(\vec{u} \otimes \vec{v})_{ij} = u_i v_j$
  \end{block}
  
  \pause
  
  \begin{block}{4. Contracción}
    Reducir orden sumando: tensor orden $(m+n-2)$ de tensores orden $m$ y $n$
  \end{block}
  
  \pause
  
  \begin{block}{5. Transformaciones}
    Cambio de coordenadas: $T'_{ij} = R_{ik} R_{jl} T_{kl}$
    
    Invariantes: traza, determinante, autovalores
  \end{block}
\end{frame}

\begin{frame}{Ideas clave para recordar}
  \begin{alertblock}{Conceptos fundamentales}
    \begin{enumerate}
      \item Tensores generalizan escalares, vectores y matrices
      \item Orden = número de índices necesarios
      \item Notación Einstein simplifica las sumas
      \item Contracción = operación fundamental (incluye mult. matriz-vector)
      \item Algunos valores no cambian con rotaciones (invariantes)
    \end{enumerate}
  \end{alertblock}
  
  \pause
  
  \begin{block}{Aplicaciones en ingeniería}
    \begin{itemize}
      \item Tensor de esfuerzos: $\sigma_{ij}$ (describe fuerzas internas)
      \item Tensor de deformación: $\varepsilon_{ij}$ (describe deformaciones)
      \item Tensor de elasticidad: $C_{ijkl}$ (relaciona esfuerzo y deformación)
      \item Análisis de estados de carga complejos
    \end{itemize}
  \end{block}
\end{frame}

\begin{frame}{¿Por qué estudiar tensores?}
  \begin{block}{Ventajas}
    \begin{enumerate}
      \item \textbf{Notación compacta}: ecuaciones simples para sistemas complejos
      \item \textbf{Independencia de coordenadas}: las leyes físicas no dependen de cómo midas
      \item \textbf{Generalización}: un framework unificado para muchos conceptos
      \item \textbf{Computación}: librerías como NumPy/PyTorch optimizadas para tensores
    \end{enumerate}
  \end{block}
  
  \pause
  
  \begin{alertblock}{En la práctica}
    No necesitas memorizar todas las fórmulas. Lo importante es:
    \begin{itemize}
      \item Entender qué representa cada tensor
      \item Saber qué operaciones existen
      \item Usar herramientas computacionales (NumPy, etc.)
    \end{itemize}
  \end{alertblock}
\end{frame}

\end{document}
