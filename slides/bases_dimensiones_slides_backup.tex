\documentclass[aspectratio=169]{beamer}
% slides/_preamble_slides.tex
% Tema moderno minimalista
\usetheme{default}
\usecolortheme{default}

\usepackage[spanish]{babel}
\usepackage[utf8]{inputenc}
\usepackage[T1]{fontenc}
\usepackage{lmodern}
\usepackage{graphicx}
\usepackage{ifthen}
\graphicspath{{figuras/}{./figuras/}{./}}

% Colores modernos
\definecolor{primaryColor}{RGB}{0, 102, 204}      % Azul profesional
\definecolor{secondaryColor}{RGB}{255, 87, 51}    % Coral/naranja moderno
\definecolor{accentColor}{RGB}{46, 204, 113}      % Verde menta
\definecolor{darkGray}{RGB}{44, 62, 80}           % Gris oscuro
\definecolor{lightGray}{RGB}{236, 240, 241}       % Gris claro
\definecolor{textColor}{RGB}{33, 37, 41}          % Negro suave

% Configuración del tema
\setbeamercolor{structure}{fg=primaryColor}
\setbeamercolor{palette primary}{bg=primaryColor,fg=white}
\setbeamercolor{palette secondary}{bg=secondaryColor,fg=white}
\setbeamercolor{palette tertiary}{bg=darkGray,fg=white}
\setbeamercolor{palette quaternary}{bg=accentColor,fg=white}

% Título y frames
\setbeamercolor{titlelike}{parent=palette primary}
\setbeamercolor{frametitle}{parent=palette primary}
\setbeamercolor{block title}{bg=primaryColor,fg=white}
\setbeamercolor{block body}{bg=lightGray,fg=textColor}
\setbeamercolor{block title alerted}{bg=secondaryColor,fg=white}
\setbeamercolor{block body alerted}{bg=lightGray,fg=textColor}
\setbeamercolor{block title example}{bg=accentColor,fg=white}
\setbeamercolor{block body example}{bg=lightGray,fg=textColor}

% Items y enumeración
\setbeamercolor{item}{fg=primaryColor}
\setbeamercolor{subitem}{fg=secondaryColor}
\setbeamercolor{subsubitem}{fg=accentColor}

% Estilo de los frames
\setbeamertemplate{frametitle}{
  \vspace{0.5em}
  \insertframetitle
  \vspace{0.3em}
}

% Bloques con esquinas redondeadas
\setbeamertemplate{blocks}[rounded][shadow=false]

% Items modernos
\setbeamertemplate{itemize items}[circle]
\setbeamertemplate{enumerate items}[circle]

% Sin símbolos de navegación
\setbeamertemplate{navigation symbols}{}

% Footline moderno - color único azul oscuro
\setbeamertemplate{footline}{
  \leavevmode%
  \hbox{%
    \begin{beamercolorbox}[wd=\paperwidth,ht=2.5ex,dp=1ex,leftskip=.3cm,rightskip=.3cm]{palette tertiary}%
      \usebeamerfont{author in head/foot}\insertshortauthor
      \hfill
      \usebeamerfont{title in head/foot}\insertshorttitle
      \hfill
      \usebeamerfont{date in head/foot}\insertshortdate{}
      \hspace*{2em}
      \insertframenumber{} / \inserttotalframenumber
    \end{beamercolorbox}}%
  \vskip0pt%
}

% Título con diseño moderno
\setbeamertemplate{title page}{
  \vbox{}
  \vfill
  \begingroup
    \centering
    \begin{beamercolorbox}[sep=8pt,center]{title}
      \usebeamerfont{title}\inserttitle\par%
      \ifx\insertsubtitle\@empty%
      \else%
        \vskip0.25em%
        {\usebeamerfont{subtitle}\usebeamercolor[fg]{subtitle}\insertsubtitle\par}%
      \fi%     
    \end{beamercolorbox}%
    \vskip1em\par
    \begin{beamercolorbox}[sep=8pt,center]{author}
      \usebeamerfont{author}\insertauthor
    \end{beamercolorbox}
    \begin{beamercolorbox}[sep=8pt,center]{institute}
      \usebeamerfont{institute}\insertinstitute
    \end{beamercolorbox}
    \begin{beamercolorbox}[sep=8pt,center]{date}
      \usebeamerfont{date}\insertdate
    \end{beamercolorbox}\vskip0.5em
  \endgroup
  \vfill
}

% Tipografía moderna
\setbeamerfont{title}{size=\Large,series=\bfseries}
\setbeamerfont{subtitle}{size=\large}
\setbeamerfont{frametitle}{size=\large,series=\bfseries}
\setbeamerfont{block title}{size=\normalsize,series=\bfseries}

\hypersetup{colorlinks=true, linkcolor=primaryColor, urlcolor=primaryColor, citecolor=secondaryColor}

% Definir entornos en español
\newtheorem{teorema}{Teorema}
\newtheorem{corolario}{Corolario}
\theoremstyle{definition}
\newtheorem{ejemplo}{Ejemplo}
\newtheorem{ejemplos}{Ejemplos}

\usepackage{amsmath, amssymb, amsthm}
\usepackage{tikz}

\title{Bases, Dimensiones y Espacios Generados}
\subtitle{Álgebra Lineal}
\author{Jordan Blancas}
\date{\today}

\begin{document}

\begin{frame}
  \titlepage
\end{frame}

\begin{frame}{Contenido}
  \tableofcontents
\end{frame}

% ========================================
% Bases, Dimensiones y Espacios Generados
% ========================================

\section{Espacios Vectoriales}

\begin{frame}{Espacios Vectoriales}
  \begin{block}{Definición}
    Un \textbf{espacio vectorial} $V$ sobre un campo $\mathbb{F}$ es un conjunto con dos operaciones:
    \begin{itemize}
      \item Suma de vectores: $+: V \times V \to V$
      \item Multiplicación por escalar: $\cdot: \mathbb{F} \times V \to V$
    \end{itemize}
    que satisfacen 8 axiomas (asociatividad, conmutatividad, etc.)
  \end{block}
  
  \begin{ejemplos}
    \begin{itemize}
      \item $\mathbb{R}^n$ con suma y multiplicación estándar
      \item Polinomios $P_n(\mathbb{R})$ de grado $\leq n$
      \item Matrices $M_{m \times n}(\mathbb{R})$
    \end{itemize}
  \end{ejemplos}
\end{frame}

\section{Combinaciones Lineales}

\begin{frame}{Combinación Lineal}
  \begin{block}{Definición}
    Sea $V$ un espacio vectorial y $\{v_1, v_2, \ldots, v_k\} \subset V$. Un vector $v \in V$ es una \textbf{combinación lineal} de $\{v_1, \ldots, v_k\}$ si:
    $$v = \alpha_1 v_1 + \alpha_2 v_2 + \cdots + \alpha_k v_k$$
    donde $\alpha_1, \alpha_2, \ldots, \alpha_k \in \mathbb{F}$
  \end{block}
  
  \begin{ejemplo}
    En $\mathbb{R}^3$: El vector $(5, 1, 7)$ es combinación lineal de $(1,0,1)$ y $(2,1,3)$:
    $$(5, 1, 7) = 1 \cdot (1,0,1) + 2 \cdot (2,1,3)$$
  \end{ejemplo}
\end{frame}

\section{Espacio Generado}

\begin{frame}{Espacio Generado}
  \begin{block}{Definición}
    El \textbf{espacio generado} (o span) por un conjunto $S = \{v_1, v_2, \ldots, v_k\} \subset V$ es:
    $$\text{span}(S) = \left\{ \sum_{i=1}^{k} \alpha_i v_i \,:\, \alpha_i \in \mathbb{F} \right\}$$
    Es decir, el conjunto de todas las combinaciones lineales de elementos de $S$.
  \end{block}
  
  \begin{block}{Propiedades}
    \begin{itemize}
      \item $\text{span}(S)$ es un subespacio vectorial de $V$
      \item $S \subseteq \text{span}(S)$
      \item Es el subespacio más pequeño que contiene a $S$
    \end{itemize}
  \end{block}
\end{frame}

\section{Independencia Lineal}

\begin{frame}{Dependencia e Independencia Lineal}
  \begin{block}{Definición: Independencia Lineal}
    Un conjunto $\{v_1, \ldots, v_k\}$ es \textbf{linealmente independiente} (LI) si:
    $$\alpha_1 v_1 + \cdots + \alpha_k v_k = 0 \implies \alpha_1 = \cdots = \alpha_k = 0$$
  \end{block}
  
  \begin{block}{Dependencia Lineal}
    Si no es LI, es \textbf{linealmente dependiente} (LD): existe al menos un coeficiente no nulo tal que la combinación lineal da cero.
  \end{block}
  
  \begin{ejemplo}
    En $\mathbb{R}^3$:
    \begin{itemize}
      \item $\{(1,0,0), (0,1,0), (0,0,1)\}$ es LI
      \item $\{(1,2,3), (2,4,6)\}$ es LD (el segundo es múltiplo del primero)
    \end{itemize}
  \end{ejemplo}
\end{frame}

\section{Bases}

\begin{frame}{Base de un Espacio Vectorial}
  \begin{block}{Definición}
    Una \textbf{base} $\mathcal{B}$ de un espacio vectorial $V$ es un conjunto de vectores que cumple:
    \begin{enumerate}
      \item Es linealmente independiente
      \item Genera todo el espacio: $\text{span}(\mathcal{B}) = V$
    \end{enumerate}
  \end{block}
  
  \begin{block}{Propiedad Fundamental}
    Todo vector $v \in V$ se puede escribir de manera \textbf{única} como combinación lineal de los vectores de la base.
  \end{block}
  
  \begin{ejemplo}
    Base canónica de $\mathbb{R}^3$:
    $$\mathcal{B} = \{(1,0,0), (0,1,0), (0,0,1)\}$$
  \end{ejemplo}
\end{frame}

\section{Dimensión}

\begin{frame}{Dimensión}
  \begin{block}{Teorema}
    Todas las bases de un espacio vectorial $V$ tienen el mismo número de elementos.
  \end{block}
  
  \begin{block}{Definición: Dimensión}
    La \textbf{dimensión} de un espacio vectorial $V$, denotada $\dim(V)$, es el número de vectores en cualquier base de $V$.
  \end{block}
  
  \begin{ejemplos}
    \begin{itemize}
      \item $\dim(\mathbb{R}^n) = n$
      \item $\dim(P_n(\mathbb{R})) = n + 1$ (polinomios de grado $\leq n$)
      \item $\dim(M_{m \times n}(\mathbb{R})) = m \cdot n$
      \item $\dim(\{0\}) = 0$ (espacio trivial)
    \end{itemize}
  \end{ejemplos}
\end{frame}

\begin{frame}{Teorema de la Dimensión para Subespacios}
  \begin{teorema}
    Sea $V$ un espacio vectorial de dimensión finita y $W$ un subespacio de $V$. Entonces:
    $$\dim(W) \leq \dim(V)$$
    y la igualdad se cumple si y solo si $W = V$.
  \end{teorema}
  
  \begin{corolario}
    En $\mathbb{R}^n$:
    \begin{itemize}
      \item Cualquier conjunto de $n$ vectores LI forma una base
      \item Cualquier conjunto de más de $n$ vectores es LD
      \item Un subespacio de dimensión $k < n$ es un "objeto" de dimensión menor
    \end{itemize}
  \end{corolario}
\end{frame}

\section{Resumen}

\begin{frame}{Resumen}
  \begin{block}{Conceptos Clave}
    \begin{itemize}
      \item \textbf{Espacio vectorial}: estructura algebraica con suma y multiplicación por escalares
      \item \textbf{Combinación lineal}: suma ponderada de vectores
      \item \textbf{Espacio generado}: todas las combinaciones lineales posibles
      \item \textbf{Independencia lineal}: vectores que no son combinaciones entre sí
      \item \textbf{Base}: conjunto L.I. que genera todo el espacio
      \item \textbf{Dimensión}: número de vectores en una base
    \end{itemize}
  \end{block}
  
  \begin{alertblock}{Propiedad Fundamental}
    Todas las bases de un espacio tienen la misma dimensión.
  \end{alertblock}
\end{frame}

\begin{frame}{Referencias}
  \begin{thebibliography}{99}
    \bibitem{strang} G. Strang, \textit{Linear Algebra and Its Applications}, 4th ed., Cengage Learning, 2006.
    
    \bibitem{axler} S. Axler, \textit{Linear Algebra Done Right}, 3rd ed., Springer, 2015.
    
    \bibitem{friedberg} S. Friedberg, A. Insel, L. Spence, \textit{Linear Algebra}, 5th ed., Pearson, 2018.
  \end{thebibliography}
\end{frame}

\begin{frame}
  \centering
  \Huge ¿Preguntas?
  
  \vspace{1cm}
  
  \Large Gracias por su atención
\end{frame}

\end{document}
