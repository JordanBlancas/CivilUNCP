\documentclass[aspectratio=169]{beamer}
% slides/_preamble_slides.tex
% Tema moderno minimalista
\usetheme{default}
\usecolortheme{default}

\usepackage[spanish]{babel}
\usepackage[utf8]{inputenc}
\usepackage[T1]{fontenc}
\usepackage{lmodern}
\usepackage{graphicx}
\usepackage{ifthen}
\graphicspath{{figuras/}{./figuras/}{./}}

% Colores modernos
\definecolor{primaryColor}{RGB}{0, 102, 204}      % Azul profesional
\definecolor{secondaryColor}{RGB}{255, 87, 51}    % Coral/naranja moderno
\definecolor{accentColor}{RGB}{46, 204, 113}      % Verde menta
\definecolor{darkGray}{RGB}{44, 62, 80}           % Gris oscuro
\definecolor{lightGray}{RGB}{236, 240, 241}       % Gris claro
\definecolor{textColor}{RGB}{33, 37, 41}          % Negro suave

% Configuración del tema
\setbeamercolor{structure}{fg=primaryColor}
\setbeamercolor{palette primary}{bg=primaryColor,fg=white}
\setbeamercolor{palette secondary}{bg=secondaryColor,fg=white}
\setbeamercolor{palette tertiary}{bg=darkGray,fg=white}
\setbeamercolor{palette quaternary}{bg=accentColor,fg=white}

% Título y frames
\setbeamercolor{titlelike}{parent=palette primary}
\setbeamercolor{frametitle}{parent=palette primary}
\setbeamercolor{block title}{bg=primaryColor,fg=white}
\setbeamercolor{block body}{bg=lightGray,fg=textColor}
\setbeamercolor{block title alerted}{bg=secondaryColor,fg=white}
\setbeamercolor{block body alerted}{bg=lightGray,fg=textColor}
\setbeamercolor{block title example}{bg=accentColor,fg=white}
\setbeamercolor{block body example}{bg=lightGray,fg=textColor}

% Items y enumeración
\setbeamercolor{item}{fg=primaryColor}
\setbeamercolor{subitem}{fg=secondaryColor}
\setbeamercolor{subsubitem}{fg=accentColor}

% Estilo de los frames
\setbeamertemplate{frametitle}{
  \vspace{0.5em}
  \insertframetitle
  \vspace{0.3em}
}

% Bloques con esquinas redondeadas
\setbeamertemplate{blocks}[rounded][shadow=false]

% Items modernos
\setbeamertemplate{itemize items}[circle]
\setbeamertemplate{enumerate items}[circle]

% Sin símbolos de navegación
\setbeamertemplate{navigation symbols}{}

% Footline moderno - color único azul oscuro
\setbeamertemplate{footline}{
  \leavevmode%
  \hbox{%
    \begin{beamercolorbox}[wd=\paperwidth,ht=2.5ex,dp=1ex,leftskip=.3cm,rightskip=.3cm]{palette tertiary}%
      \usebeamerfont{author in head/foot}\insertshortauthor
      \hfill
      \usebeamerfont{title in head/foot}\insertshorttitle
      \hfill
      \usebeamerfont{date in head/foot}\insertshortdate{}
      \hspace*{2em}
      \insertframenumber{} / \inserttotalframenumber
    \end{beamercolorbox}}%
  \vskip0pt%
}

% Título con diseño moderno
\setbeamertemplate{title page}{
  \vbox{}
  \vfill
  \begingroup
    \centering
    \begin{beamercolorbox}[sep=8pt,center]{title}
      \usebeamerfont{title}\inserttitle\par%
      \ifx\insertsubtitle\@empty%
      \else%
        \vskip0.25em%
        {\usebeamerfont{subtitle}\usebeamercolor[fg]{subtitle}\insertsubtitle\par}%
      \fi%     
    \end{beamercolorbox}%
    \vskip1em\par
    \begin{beamercolorbox}[sep=8pt,center]{author}
      \usebeamerfont{author}\insertauthor
    \end{beamercolorbox}
    \begin{beamercolorbox}[sep=8pt,center]{institute}
      \usebeamerfont{institute}\insertinstitute
    \end{beamercolorbox}
    \begin{beamercolorbox}[sep=8pt,center]{date}
      \usebeamerfont{date}\insertdate
    \end{beamercolorbox}\vskip0.5em
  \endgroup
  \vfill
}

% Tipografía moderna
\setbeamerfont{title}{size=\Large,series=\bfseries}
\setbeamerfont{subtitle}{size=\large}
\setbeamerfont{frametitle}{size=\large,series=\bfseries}
\setbeamerfont{block title}{size=\normalsize,series=\bfseries}

\hypersetup{colorlinks=true, linkcolor=primaryColor, urlcolor=primaryColor, citecolor=secondaryColor}

% Definir entornos en español
\newtheorem{teorema}{Teorema}
\newtheorem{corolario}{Corolario}
\theoremstyle{definition}
\newtheorem{ejemplo}{Ejemplo}
\newtheorem{ejemplos}{Ejemplos}

\usepackage{amsmath, amssymb, amsthm}
\usepackage{tikz}

\title{Transformaciones Lineales}
\subtitle{Núcleo e Imagen}
\author{Jordan Blancas}
\date{\today}

\begin{document}

\begin{frame}
  \titlepage
\end{frame}

\begin{frame}{Contenido}
  \tableofcontents
\end{frame}

% ========================================
% Transformaciones Lineales
% ========================================

\section{Transformaciones Lineales}

\begin{frame}{Transformación Lineal}
  \begin{block}{Definición}
    Sean $V$ y $W$ espacios vectoriales sobre $\mathbb{F}$. Una función $T: V \to W$ es una \textbf{transformación lineal} si:
    \begin{enumerate}
      \item $T(u + v) = T(u) + T(v)$ para todo $u, v \in V$
      \item $T(\alpha v) = \alpha T(v)$ para todo $\alpha \in \mathbb{F}$ y $v \in V$
    \end{enumerate}
  \end{block}
  
  \begin{block}{Equivalentemente}
    $T(\alpha u + \beta v) = \alpha T(u) + \beta T(v)$ para todo $u, v \in V$ y $\alpha, \beta \in \mathbb{F}$
  \end{block}
\end{frame}

\begin{frame}{Ejemplos de Transformaciones Lineales}
  \begin{ejemplos}
    \begin{enumerate}
      \item \textbf{Identidad}: $I: V \to V$, $I(v) = v$
      
      \item \textbf{Transformación cero}: $0: V \to W$, $0(v) = 0_W$
      
      \item \textbf{Multiplicación por matriz}: $T: \mathbb{R}^n \to \mathbb{R}^m$
      $$T(x) = Ax$$
      donde $A \in M_{m \times n}(\mathbb{R})$
      
      \item \textbf{Derivada}: $D: P_n(\mathbb{R}) \to P_{n-1}(\mathbb{R})$
      $$D(p(x)) = p'(x)$$
      
      \item \textbf{Integral}: $I: C[a,b] \to \mathbb{R}$
      $$I(f) = \int_a^b f(x) \, dx$$
    \end{enumerate}
  \end{ejemplos}
\end{frame}

\begin{frame}{Propiedades de Transformaciones Lineales}
  \begin{teorema}
    Sea $T: V \to W$ una transformación lineal. Entonces:
    \begin{enumerate}
      \item $T(0_V) = 0_W$
      \item $T(-v) = -T(v)$ para todo $v \in V$
      \item $T$ preserva combinaciones lineales:
      $$T\left(\sum_{i=1}^k \alpha_i v_i\right) = \sum_{i=1}^k \alpha_i T(v_i)$$
    \end{enumerate}
  \end{teorema}
  
  \begin{alertblock}{Importante}
    Una transformación lineal queda completamente determinada por su acción sobre una base de $V$.
  \end{alertblock}
\end{frame}

\section{Núcleo}

\begin{frame}{Núcleo}
  \begin{block}{Definición}
    El \textbf{núcleo} de una transformación lineal $T: V \to W$ es:
    $$\ker(T) = \{v \in V \,:\, T(v) = 0_W\}$$
    Es el conjunto de todos los vectores que $T$ mapea al vector cero.
  \end{block}
  
  \begin{teorema}
    $\ker(T)$ es un subespacio vectorial de $V$.
  \end{teorema}
  
  \begin{ejemplo}
    Para $T: \mathbb{R}^3 \to \mathbb{R}^2$, $T(x,y,z) = (x+y, 2x+2y)$:
    $$\ker(T) = \{(x,y,z) : x+y=0\} = \{(-t, t, s) : t, s \in \mathbb{R}\}$$
    Es un plano que pasa por el origen.
  \end{ejemplo}
\end{frame}

\section{Imagen}

\begin{frame}{Imagen}
  \begin{block}{Definición}
    La \textbf{imagen} (o rango) de una transformación lineal $T: V \to W$ es:
    $$\text{Im}(T) = \{w \in W \,:\, \exists v \in V \text{ tal que } T(v) = w\}$$
    Es el conjunto de todos los vectores en $W$ que son imagen de algún vector en $V$.
  \end{block}
  
  \begin{teorema}
    $\text{Im}(T)$ es un subespacio vectorial de $W$.
  \end{teorema}
  
  \begin{block}{Observación}
    Si $\{v_1, \ldots, v_n\}$ es una base de $V$, entonces:
    $$\text{Im}(T) = \text{span}\{T(v_1), \ldots, T(v_n)\}$$
  \end{block}
\end{frame}

\section{Inyectividad y Sobreyectividad}

\begin{frame}{Inyectividad y Sobreyectividad}
  \begin{block}{Transformación Inyectiva}
    $T$ es \textbf{inyectiva} si:
    $$T(u) = T(v) \implies u = v$$
    
    \textbf{Equivalentemente}: $\ker(T) = \{0\}$
  \end{block}
  
  \begin{block}{Transformación Sobreyectiva}
    $T$ es \textbf{sobreyectiva} si:
    $$\text{Im}(T) = W$$
    
    Es decir, todo vector en $W$ es imagen de algún vector en $V$.
  \end{block}
  
  \begin{block}{Isomorfismo}
    $T$ es un \textbf{isomorfismo} si es inyectiva y sobreyectiva.
  \end{block}
\end{frame}

\section{Teorema de la Dimensión}

\begin{frame}{Teorema de la Dimensión}
  \begin{teorema}[Teorema Fundamental]
    Sea $T: V \to W$ una transformación lineal, con $\dim(V) < \infty$. Entonces:
    $$\boxed{\dim(V) = \dim(\ker(T)) + \dim(\text{Im}(T))}$$
  \end{teorema}
  
  \begin{block}{Nomenclatura}
    \begin{itemize}
      \item \textbf{Nulidad}: $\text{nul}(T) = \dim(\ker(T))$
      \item \textbf{Rango}: $\text{rango}(T) = \dim(\text{Im}(T))$
    \end{itemize}
    $$\boxed{\dim(V) = \text{nul}(T) + \text{rango}(T)}$$
  \end{block}
  
  \begin{alertblock}{Consecuencia}
    Para $T: \mathbb{R}^n \to \mathbb{R}^m$ dada por $T(x) = Ax$:
    $$n = \text{nul}(A) + \text{rango}(A)$$
  \end{alertblock}
\end{frame}

\begin{frame}{Aplicaciones del Teorema}
  \begin{block}{Consecuencias Importantes}
    \begin{itemize}
      \item Si $T: V \to W$ es inyectiva, entonces $\dim(V) \leq \dim(W)$
      \item Si $T: V \to W$ es sobreyectiva, entonces $\dim(V) \geq \dim(W)$
      \item Si $T: V \to W$ es un isomorfismo, entonces $\dim(V) = \dim(W)$
    \end{itemize}
  \end{block}
  
  \begin{ejemplo}
    Sea $T: \mathbb{R}^5 \to \mathbb{R}^3$ con $\text{rank}(T) = 2$. Entonces:
    $$\text{nullity}(T) = 5 - 2 = 3$$
    El núcleo es un subespacio de dimensión 3 en $\mathbb{R}^5$.
  \end{ejemplo}
\end{frame}

\section{Representación Matricial}

\begin{frame}{Matriz de una Transformación Lineal}
  \begin{block}{Representación Matricial}
    Sea $T: \mathbb{R}^n \to \mathbb{R}^m$ lineal. Existe una única matriz $A \in M_{m \times n}(\mathbb{R})$ tal que:
    $$T(x) = Ax \quad \forall x \in \mathbb{R}^n$$
  \end{block}
  
  \begin{block}{Construcción}
    Si $\{e_1, \ldots, e_n\}$ es la base canónica de $\mathbb{R}^n$, entonces la matriz $A$ tiene como columnas:
    $$A = [T(e_1) \,|\, T(e_2) \,|\, \cdots \,|\, T(e_n)]$$
  \end{block}
  
  \begin{ejemplo}
    Para $T: \mathbb{R}^2 \to \mathbb{R}^2$, rotación por $90^\circ$:
    $$T(1,0) = (0,1), \quad T(0,1) = (-1,0)$$
    $$A = \begin{pmatrix} 0 & -1 \\ 1 & 0 \end{pmatrix}$$
  \end{ejemplo}
\end{frame}

\section{Resumen}

\begin{frame}{Resumen}
  \begin{block}{Conceptos Clave}
    \begin{itemize}
      \item \textbf{Transformación lineal}: preserva suma y multiplicación por escalares
      \item \textbf{Núcleo}: vectores que van al cero
      \item \textbf{Imagen}: vectores alcanzables
      \item \textbf{Inyectividad}: $\ker(T) = \{0\}$
      \item \textbf{Sobreyectividad}: $\text{Im}(T) = W$
    \end{itemize}
  \end{block}
  
  \begin{alertblock}{Teorema de la Dimensión}
    $$\dim(V) = \dim(\ker T) + \dim(\text{Im} T)$$
    Es la relación fundamental entre el dominio, núcleo e imagen.
  \end{alertblock}
\end{frame}

\begin{frame}{Referencias}
  \begin{thebibliography}{99}
    \bibitem{strang} G. Strang, \textit{Linear Algebra and Its Applications}, 4th ed., Cengage Learning, 2006.
    
    \bibitem{axler} S. Axler, \textit{Linear Algebra Done Right}, 3rd ed., Springer, 2015.
    
    \bibitem{friedberg} S. Friedberg, A. Insel, L. Spence, \textit{Linear Algebra}, 5th ed., Pearson, 2018.
  \end{thebibliography}
\end{frame}

\begin{frame}
  \centering
  \Huge ¿Preguntas?
  
  \vspace{1cm}
  
  \Large Gracias por su atención
\end{frame}

\end{document}
