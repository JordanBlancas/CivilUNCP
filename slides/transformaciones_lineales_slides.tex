\documentclass[aspectratio=169]{beamer}
% slides/_preamble_slides.tex
% Tema moderno minimalista
\usetheme{default}
\usecolortheme{default}

\usepackage[spanish]{babel}
\usepackage[utf8]{inputenc}
\usepackage[T1]{fontenc}
\usepackage{lmodern}
\usepackage{graphicx}
\usepackage{ifthen}
\graphicspath{{figuras/}{./figuras/}{./}}

% Colores modernos
\definecolor{primaryColor}{RGB}{0, 102, 204}      % Azul profesional
\definecolor{secondaryColor}{RGB}{255, 87, 51}    % Coral/naranja moderno
\definecolor{accentColor}{RGB}{46, 204, 113}      % Verde menta
\definecolor{darkGray}{RGB}{44, 62, 80}           % Gris oscuro
\definecolor{lightGray}{RGB}{236, 240, 241}       % Gris claro
\definecolor{textColor}{RGB}{33, 37, 41}          % Negro suave

% Configuración del tema
\setbeamercolor{structure}{fg=primaryColor}
\setbeamercolor{palette primary}{bg=primaryColor,fg=white}
\setbeamercolor{palette secondary}{bg=secondaryColor,fg=white}
\setbeamercolor{palette tertiary}{bg=darkGray,fg=white}
\setbeamercolor{palette quaternary}{bg=accentColor,fg=white}

% Título y frames
\setbeamercolor{titlelike}{parent=palette primary}
\setbeamercolor{frametitle}{parent=palette primary}
\setbeamercolor{block title}{bg=primaryColor,fg=white}
\setbeamercolor{block body}{bg=lightGray,fg=textColor}
\setbeamercolor{block title alerted}{bg=secondaryColor,fg=white}
\setbeamercolor{block body alerted}{bg=lightGray,fg=textColor}
\setbeamercolor{block title example}{bg=accentColor,fg=white}
\setbeamercolor{block body example}{bg=lightGray,fg=textColor}

% Items y enumeración
\setbeamercolor{item}{fg=primaryColor}
\setbeamercolor{subitem}{fg=secondaryColor}
\setbeamercolor{subsubitem}{fg=accentColor}

% Estilo de los frames
\setbeamertemplate{frametitle}{
  \vspace{0.5em}
  \insertframetitle
  \vspace{0.3em}
}

% Bloques con esquinas redondeadas
\setbeamertemplate{blocks}[rounded][shadow=false]

% Items modernos
\setbeamertemplate{itemize items}[circle]
\setbeamertemplate{enumerate items}[circle]

% Sin símbolos de navegación
\setbeamertemplate{navigation symbols}{}

% Footline moderno - color único azul oscuro
\setbeamertemplate{footline}{
  \leavevmode%
  \hbox{%
    \begin{beamercolorbox}[wd=\paperwidth,ht=2.5ex,dp=1ex,leftskip=.3cm,rightskip=.3cm]{palette tertiary}%
      \usebeamerfont{author in head/foot}\insertshortauthor
      \hfill
      \usebeamerfont{title in head/foot}\insertshorttitle
      \hfill
      \usebeamerfont{date in head/foot}\insertshortdate{}
      \hspace*{2em}
      \insertframenumber{} / \inserttotalframenumber
    \end{beamercolorbox}}%
  \vskip0pt%
}

% Título con diseño moderno
\setbeamertemplate{title page}{
  \vbox{}
  \vfill
  \begingroup
    \centering
    \begin{beamercolorbox}[sep=8pt,center]{title}
      \usebeamerfont{title}\inserttitle\par%
      \ifx\insertsubtitle\@empty%
      \else%
        \vskip0.25em%
        {\usebeamerfont{subtitle}\usebeamercolor[fg]{subtitle}\insertsubtitle\par}%
      \fi%     
    \end{beamercolorbox}%
    \vskip1em\par
    \begin{beamercolorbox}[sep=8pt,center]{author}
      \usebeamerfont{author}\insertauthor
    \end{beamercolorbox}
    \begin{beamercolorbox}[sep=8pt,center]{institute}
      \usebeamerfont{institute}\insertinstitute
    \end{beamercolorbox}
    \begin{beamercolorbox}[sep=8pt,center]{date}
      \usebeamerfont{date}\insertdate
    \end{beamercolorbox}\vskip0.5em
  \endgroup
  \vfill
}

% Tipografía moderna
\setbeamerfont{title}{size=\Large,series=\bfseries}
\setbeamerfont{subtitle}{size=\large}
\setbeamerfont{frametitle}{size=\large,series=\bfseries}
\setbeamerfont{block title}{size=\normalsize,series=\bfseries}

\hypersetup{colorlinks=true, linkcolor=primaryColor, urlcolor=primaryColor, citecolor=secondaryColor}

% Definir entornos en español
\newtheorem{teorema}{Teorema}
\newtheorem{corolario}{Corolario}
\theoremstyle{definition}
\newtheorem{ejemplo}{Ejemplo}
\newtheorem{ejemplos}{Ejemplos}

\usepackage{amsmath, amssymb, amsthm}
\usepackage{listings}

\lstset{
  language=Python,
  basicstyle=\ttfamily\scriptsize,
  keywordstyle=\color{blue},
  commentstyle=\color{green!60!black},
  stringstyle=\color{red},
  showstringspaces=false,
  breaklines=true,
  frame=single,
  numbers=left,
  numberstyle=\tiny\color{gray},
  backgroundcolor=\color{lightGray}
}

\title{Transformaciones Lineales}
\subtitle{Álgebra Lineal}
\author{Jordan Blancas}
\date{\today}

\begin{document}

\begin{frame}
  \titlepage
\end{frame}

\section{Motivación}
\begin{frame}{¿Por qué estudiar transformaciones lineales?}
  \begin{block}{Ejemplo visual}
    Rotar una figura, proyectar una sombra, escalar una imagen, calcular fuerzas a partir de desplazamientos...
  \end{block}
  \pause
  \begin{alertblock}{Idea}
    Todas estas acciones son funciones que transforman vectores en otros vectores.
  \end{alertblock}
\end{frame}

\section{Ejemplo concreto}
\begin{frame}{Ejemplo numérico antes de la definición}
  \begin{block}{Transformación simple}
    Sea $T: \mathbb{R}^2 \to \mathbb{R}^2$ definida por:
    $$T(x, y) = (2x, 3y)$$
    \begin{itemize}
      \item $T(1, 0) = (2, 0)$
      \item $T(0, 1) = (0, 3)$
      \item $T(2, 5) = (4, 15)$
    \end{itemize}
  \end{block}
  \pause
  \begin{block}{¿Qué pasa si sumo y luego transformo?}
    $T(2+3, 1+2) = T(5, 3) = (10, 9)$\\
    $T(2, 1) + T(3, 2) = (4, 3) + (6, 6) = (10, 9)$
  \end{block}
\end{frame}

\section{Definición}
\begin{frame}{Definición formal}
  \begin{block}{Definición}
    Una función $T: V \to W$ es \textbf{lineal} si cumple:
    \begin{enumerate}
      \item $T(\vec{u} + \vec{v}) = T(\vec{u}) + T(\vec{v})$
      \item $T(\alpha \vec{v}) = \alpha T(\vec{v})$
    \end{enumerate}
  \end{block}
  \pause
  \begin{alertblock}{Consecuencia}
    $T(\vec{0}) = \vec{0}$ siempre
  \end{alertblock}
\end{frame}

\begin{frame}{Ejemplos y contraejemplos}
  \begin{block}{Lineales $\checkmark$}
    \begin{itemize}
      \item $T(x, y) = (2x, 3y)$
      \item $T(x, y) = (x-y, x+y)$
      \item $T(\vec{x}) = A\vec{x}$
    \end{itemize}
  \end{block}
  \pause
  \begin{alertblock}{NO lineales $\times$}
    \begin{itemize}
      \item $T(x, y) = (x+1, y)$
      \item $T(x, y) = (x^2, y)$
      \item $T(x, y) = (\sin(x), y)$
    \end{itemize}
  \end{alertblock}
\end{frame}

\section{Matriz asociada}
\begin{frame}{¿Cómo se representa una transformación?}
  \begin{block}{Idea clave}
    Si sabes qué le pasa a los vectores de la base, sabes qué le pasa a cualquier vector.
  \end{block}
  \pause
  \begin{block}{Ejemplo}
    En $\mathbb{R}^2$, cualquier $(x, y) = x(1,0) + y(0,1)$\\
    Si $T(1,0) = (a, c)$ y $T(0,1) = (b, d)$, entonces:
    $$T(x,y) = x(a,c) + y(b,d) = (ax+by, cx+dy)$$
    $$T\begin{pmatrix} x \\ y \end{pmatrix} = \begin{pmatrix} a & b \\ c & d \end{pmatrix} \begin{pmatrix} x \\ y \end{pmatrix}$$
  \end{block}
\end{frame}

\section{Núcleo e Imagen}
\begin{frame}{¿Qué se pierde y qué se alcanza?}
  \begin{block}{Núcleo (Kernel)}
    $\text{Ker}(T) = \{\vec{v} \in V : T(\vec{v}) = \vec{0}\}$\\
    Vectores que se mapean a cero. Si solo el cero va a cero, $T$ es inyectiva.
  \end{block}
  \pause
  \begin{block}{Imagen}
    $\text{Im}(T) = \{T(\vec{v}) : \vec{v} \in V\}$\\
    Vectores que se pueden alcanzar. Si se alcanza todo el codominio, $T$ es sobreyectiva.
  \end{block}
  \pause
  \begin{ejemplo}
    Proyección $T(x, y) = (x, 0)$:
    $\text{Ker}(T) = \{(0, y)\}$ (eje $y$), $\text{Im}(T) = \{(x, 0)\}$ (eje $x$)
  \end{ejemplo}
\end{frame}

\section{Teorema}
\begin{frame}{Teorema de la Dimensión}
  \begin{teorema}[Fundamental]
    Para $T: V \to W$ lineal:
    $$\boxed{\dim(V) = \dim(\text{Ker}(T)) + \dim(\text{Im}(T))}$$
  \end{teorema}
  \pause
  \begin{block}{Interpretación}
    Lo que se pierde (núcleo) + lo que se alcanza (imagen) = dimensión total
  \end{block}
  \pause
  \begin{ejemplo}
    $T: \mathbb{R}^5 \to \mathbb{R}^3$ con $\dim(\text{Ker}(T)) = 2$\\
    $\Rightarrow \dim(\text{Im}(T)) = 5 - 2 = 3$\\
    $\Rightarrow$ $T$ es sobreyectiva
  \end{ejemplo}
\end{frame}

\section{Tipos}
\begin{frame}{Tipos de Transformaciones}
  \begin{block}{Inyectiva (uno a uno)}
    $\text{Ker}(T) = \{\vec{0}\}$: no se pierde información
  \end{block}
  \begin{block}{Sobreyectiva (sobre)}
    $\text{Im}(T) = W$: se alcanza todo el codominio
  \end{block}
  \begin{block}{Biyectiva (Isomorfismo)}
    Inyectiva y sobreyectiva: existe $T^{-1}$\\
    Para $T: \mathbb{R}^n \to \mathbb{R}^n$: $\det(A) \neq 0$
  \end{block}
\end{frame}

\section{Código}
\begin{frame}[fragile]{Código Python básico}
\begin{lstlisting}
import numpy as np

# Matriz de transformacion
A = np.array([[2, 1], [1, 3]])
v = np.array([1, 2])
print("T(v) =", A @ v)

# Nucleo e imagen
rango = np.linalg.matrix_rank(A)
n = A.shape[1]
nulidad = n - rango
print(f"dim(Im(T)) = {rango}")
print(f"dim(Ker(T)) = {nulidad}")
print(f"Teorema: {n} = {nulidad} + {rango}")
\end{lstlisting}
\end{frame}

\section{Resumen}
\begin{frame}{Resumen y aplicaciones}
  \begin{enumerate}
    \item Motivación concreta
    \item Ejemplo antes de la definición
    \item Definición formal
    \item Ejemplos y contraejemplos
    \item Matriz asociada
    \item Núcleo e imagen
    \item Teorema de la dimensión
    \item Tipos de transformaciones
    \item Código Python
  \end{enumerate}
  \begin{alertblock}{Aplicaciones}
    \begin{itemize}
      \item Rotaciones, proyecciones, sistemas $F=Ku$
      \item Compresión de datos (PCA, SVD)
      \item Resolución de sistemas lineales
    \end{itemize}
  \end{alertblock}
\end{frame}

\end{document}
