\documentclass[aspectratio=169]{beamer}
% slides/_preamble_slides.tex
% Tema moderno minimalista
\usetheme{default}
\usecolortheme{default}

\usepackage[spanish]{babel}
\usepackage[utf8]{inputenc}
\usepackage[T1]{fontenc}
\usepackage{lmodern}
\usepackage{graphicx}
\usepackage{ifthen}
\graphicspath{{figuras/}{./figuras/}{./}}

% Colores modernos
\definecolor{primaryColor}{RGB}{0, 102, 204}      % Azul profesional
\definecolor{secondaryColor}{RGB}{255, 87, 51}    % Coral/naranja moderno
\definecolor{accentColor}{RGB}{46, 204, 113}      % Verde menta
\definecolor{darkGray}{RGB}{44, 62, 80}           % Gris oscuro
\definecolor{lightGray}{RGB}{236, 240, 241}       % Gris claro
\definecolor{textColor}{RGB}{33, 37, 41}          % Negro suave

% Configuración del tema
\setbeamercolor{structure}{fg=primaryColor}
\setbeamercolor{palette primary}{bg=primaryColor,fg=white}
\setbeamercolor{palette secondary}{bg=secondaryColor,fg=white}
\setbeamercolor{palette tertiary}{bg=darkGray,fg=white}
\setbeamercolor{palette quaternary}{bg=accentColor,fg=white}

% Título y frames
\setbeamercolor{titlelike}{parent=palette primary}
\setbeamercolor{frametitle}{parent=palette primary}
\setbeamercolor{block title}{bg=primaryColor,fg=white}
\setbeamercolor{block body}{bg=lightGray,fg=textColor}
\setbeamercolor{block title alerted}{bg=secondaryColor,fg=white}
\setbeamercolor{block body alerted}{bg=lightGray,fg=textColor}
\setbeamercolor{block title example}{bg=accentColor,fg=white}
\setbeamercolor{block body example}{bg=lightGray,fg=textColor}

% Items y enumeración
\setbeamercolor{item}{fg=primaryColor}
\setbeamercolor{subitem}{fg=secondaryColor}
\setbeamercolor{subsubitem}{fg=accentColor}

% Estilo de los frames
\setbeamertemplate{frametitle}{
  \vspace{0.5em}
  \insertframetitle
  \vspace{0.3em}
}

% Bloques con esquinas redondeadas
\setbeamertemplate{blocks}[rounded][shadow=false]

% Items modernos
\setbeamertemplate{itemize items}[circle]
\setbeamertemplate{enumerate items}[circle]

% Sin símbolos de navegación
\setbeamertemplate{navigation symbols}{}

% Footline moderno - color único azul oscuro
\setbeamertemplate{footline}{
  \leavevmode%
  \hbox{%
    \begin{beamercolorbox}[wd=\paperwidth,ht=2.5ex,dp=1ex,leftskip=.3cm,rightskip=.3cm]{palette tertiary}%
      \usebeamerfont{author in head/foot}\insertshortauthor
      \hfill
      \usebeamerfont{title in head/foot}\insertshorttitle
      \hfill
      \usebeamerfont{date in head/foot}\insertshortdate{}
      \hspace*{2em}
      \insertframenumber{} / \inserttotalframenumber
    \end{beamercolorbox}}%
  \vskip0pt%
}

% Título con diseño moderno
\setbeamertemplate{title page}{
  \vbox{}
  \vfill
  \begingroup
    \centering
    \begin{beamercolorbox}[sep=8pt,center]{title}
      \usebeamerfont{title}\inserttitle\par%
      \ifx\insertsubtitle\@empty%
      \else%
        \vskip0.25em%
        {\usebeamerfont{subtitle}\usebeamercolor[fg]{subtitle}\insertsubtitle\par}%
      \fi%     
    \end{beamercolorbox}%
    \vskip1em\par
    \begin{beamercolorbox}[sep=8pt,center]{author}
      \usebeamerfont{author}\insertauthor
    \end{beamercolorbox}
    \begin{beamercolorbox}[sep=8pt,center]{institute}
      \usebeamerfont{institute}\insertinstitute
    \end{beamercolorbox}
    \begin{beamercolorbox}[sep=8pt,center]{date}
      \usebeamerfont{date}\insertdate
    \end{beamercolorbox}\vskip0.5em
  \endgroup
  \vfill
}

% Tipografía moderna
\setbeamerfont{title}{size=\Large,series=\bfseries}
\setbeamerfont{subtitle}{size=\large}
\setbeamerfont{frametitle}{size=\large,series=\bfseries}
\setbeamerfont{block title}{size=\normalsize,series=\bfseries}

\hypersetup{colorlinks=true, linkcolor=primaryColor, urlcolor=primaryColor, citecolor=secondaryColor}

% Definir entornos en español
\newtheorem{teorema}{Teorema}
\newtheorem{corolario}{Corolario}
\theoremstyle{definition}
\newtheorem{ejemplo}{Ejemplo}
\newtheorem{ejemplos}{Ejemplos}

\usepackage{amsmath, amssymb, amsthm}
\usepackage{tikz}

\title{Tensores}
\subtitle{Álgebra Multilineal}
\author{Jordan Blancas}
\date{\today}

\begin{document}

\begin{frame}
  \titlepage
\end{frame}

\begin{frame}{Contenido}
  \tableofcontents
\end{frame}

% ========================================
% Tensores
% ========================================

\section{Introducción a los Tensores}

\begin{frame}{¿Qué es un Tensor?}
  \begin{block}{Intuición}
    Un tensor es una generalización multidimensional de:
    \begin{itemize}
      \item Escalar (tensor de orden 0): un número
      \item Vector (tensor de orden 1): lista de números
      \item Matriz (tensor de orden 2): tabla de números
      \item Tensor de orden $n$: arreglo $n$-dimensional de números
    \end{itemize}
  \end{block}
  
  \begin{block}{Definición Informal}
    Un tensor de orden $n$ (o rango $n$) en un espacio vectorial $V$ de dimensión $d$ es un objeto con $d^n$ componentes que se transforma de cierta manera bajo cambios de base.
  \end{block}
\end{frame}

\section{Notación de Índices}

\begin{frame}{Notación de Índices}
  \begin{block}{Convención}
    Un tensor se denota con índices superiores e inferiores:
    $$T^{i_1 i_2 \cdots i_p}_{j_1 j_2 \cdots j_q}$$
    \begin{itemize}
      \item Índices superiores: componentes \textbf{contravariantes}
      \item Índices inferiores: componentes \textbf{covariantes}
      \item El tensor es de tipo $(p, q)$
    \end{itemize}
  \end{block}
  
  \begin{ejemplos}
    \begin{itemize}
      \item $v^i$: vector (tensor tipo $(1,0)$)
      \item $\omega_j$: covector o 1-forma (tensor tipo $(0,1)$)
      \item $A^i_j$: matriz o endomorfismo (tensor tipo $(1,1)$)
      \item $T^{ij}_{k}$: tensor tipo $(2,1)$
    \end{itemize}
  \end{ejemplos}
\end{frame}

\section{Producto Tensorial}

\begin{frame}{Producto Tensorial}
  \begin{block}{Definición}
    Sean $V$ y $W$ espacios vectoriales. El \textbf{producto tensorial} $V \otimes W$ es un nuevo espacio vectorial generado por símbolos $v \otimes w$ con $v \in V$, $w \in W$, sujeto a:
    \begin{align*}
      (v_1 + v_2) \otimes w &= v_1 \otimes w + v_2 \otimes w \\
      v \otimes (w_1 + w_2) &= v \otimes w_1 + v \otimes w_2 \\
      (\alpha v) \otimes w &= v \otimes (\alpha w) = \alpha(v \otimes w)
    \end{align*}
  \end{block}
  
  \begin{block}{Dimensión}
    Si $\dim(V) = n$ y $\dim(W) = m$, entonces:
    $$\dim(V \otimes W) = n \cdot m$$
  \end{block}
\end{frame}

\begin{frame}{Bases del Producto Tensorial}
  \begin{block}{Teorema}
    Si $\{e_1, \ldots, e_n\}$ es base de $V$ y $\{f_1, \ldots, f_m\}$ es base de $W$, entonces:
    $$\{e_i \otimes f_j \,:\, 1 \leq i \leq n, \, 1 \leq j \leq m\}$$
    es una base de $V \otimes W$.
  \end{block}
  
  \begin{ejemplo}
    En $\mathbb{R}^2 \otimes \mathbb{R}^2$, con bases canónicas:
    \begin{align*}
      &e_1 \otimes e_1, \quad e_1 \otimes e_2, \\
      &e_2 \otimes e_1, \quad e_2 \otimes e_2
    \end{align*}
    forman una base. Un tensor general es:
    $$T = T^{11} e_1 \otimes e_1 + T^{12} e_1 \otimes e_2 + T^{21} e_2 \otimes e_1 + T^{22} e_2 \otimes e_2$$
  \end{ejemplo}
\end{frame}

\section{Tensores como Aplicaciones Multilineales}

\begin{frame}{Tensores como Aplicaciones Multilineales}
  \begin{block}{Definición Alternativa}
    Un tensor de tipo $(p, q)$ sobre un espacio $V$ es una aplicación multilineal:
    $$T: \underbrace{V^* \times \cdots \times V^*}_{p \text{ veces}} \times \underbrace{V \times \cdots \times V}_{q \text{ veces}} \to \mathbb{F}$$
    donde $V^*$ es el espacio dual de $V$.
  \end{block}
  
  \begin{ejemplos}
    \begin{itemize}
      \item Tensor $(0,0)$: escalar
      \item Tensor $(1,0)$: vector en $V$
      \item Tensor $(0,1)$: forma lineal (elemento de $V^*$)
      \item Tensor $(1,1)$: transformación lineal $V \to V$
      \item Tensor $(0,2)$: forma bilineal (ej: producto interno)
    \end{itemize}
  \end{ejemplos}
\end{frame}

\section{Operaciones con Tensores}

\begin{frame}{Contracción de Tensores}
  \begin{block}{Definición}
    La \textbf{contracción} es una operación que reduce el rango de un tensor igualando un índice superior con uno inferior y sumando:
    $$T^{i_1 \cdots i_p}_{j_1 \cdots j_q} \xrightarrow{\text{contracción}} S^{i_1 \cdots \hat{i}_k \cdots i_p}_{j_1 \cdots \hat{j}_\ell \cdots j_q} = \sum_{m} T^{i_1 \cdots m \cdots i_p}_{j_1 \cdots m \cdots j_q}$$
    donde $\hat{i}_k$ y $\hat{j}_\ell$ indican índices omitidos.
  \end{block}
  
  \begin{ejemplo}
    Para un tensor $T^i_j$ (matriz), la contracción da:
    $$T^i_i = \sum_{i=1}^n T^i_i = \text{tr}(T)$$
    ¡Es la traza de la matriz!
  \end{ejemplo}
\end{frame}

\begin{frame}{Operaciones con Tensores}
  \begin{block}{Operaciones Básicas}
    \begin{enumerate}
      \item \textbf{Suma}: Solo entre tensores del mismo tipo
      $$T + S = (T^{i_1\cdots i_p}_{j_1\cdots j_q} + S^{i_1\cdots i_p}_{j_1\cdots j_q})$$
      
      \item \textbf{Producto tensorial}: Aumenta el rango
      $$T \otimes S$$
      
      \item \textbf{Contracción}: Disminuye el rango
      
      \item \textbf{Transformación de coordenadas}: Bajo cambio de base $e_i' = A^j_i e_j$:
      $$T'^{i_1\cdots}_{j_1\cdots} = A^{i_1}_{k_1} \cdots (A^{-1})^{\ell_1}_{j_1} \cdots T^{k_1\cdots}_{\ell_1\cdots}$$
    \end{enumerate}
  \end{block}
\end{frame}

\section{Aplicaciones de Tensores}

\begin{frame}{Aplicaciones de Tensores}
  \begin{block}{En Física}
    \begin{itemize}
      \item \textbf{Mecánica}: Tensor de inercia, tensor de esfuerzos
      \item \textbf{Relatividad General}: Tensor métrico $g_{\mu\nu}$, tensor de curvatura de Riemann $R^\rho_{\sigma\mu\nu}$
      \item \textbf{Electromagnetismo}: Tensor electromagnético $F^{\mu\nu}$
    \end{itemize}
  \end{block}
  
  \begin{block}{En Ingeniería y Ciencias}
    \begin{itemize}
      \item \textbf{Análisis de datos}: Descomposición tensorial (Tucker, PARAFAC)
      \item \textbf{Machine Learning}: Redes neuronales tensoriales, TensorFlow
      \item \textbf{Procesamiento de señales}: Análisis multilineal
      \item \textbf{Mecánica de medios continuos}: Tensor de deformación, tensor de Cauchy
    \end{itemize}
  \end{block}
\end{frame}

\section{Ejemplo: Tensor Métrico}

\begin{frame}{Ejemplo: Tensor Métrico}
  \begin{block}{Definición}
    En geometría diferencial, el \textbf{tensor métrico} $g$ es un tensor simétrico de tipo $(0,2)$ que define el producto interno:
    $$g: V \times V \to \mathbb{R}$$
    $$g(v, w) = g_{ij} v^i w^j$$
  \end{block}
  
  \begin{ejemplo}
    En $\mathbb{R}^3$ con coordenadas cartesianas:
    $$g_{ij} = \begin{pmatrix} 1 & 0 & 0 \\ 0 & 1 & 0 \\ 0 & 0 & 1 \end{pmatrix}$$
    
    En coordenadas esféricas $(r, \theta, \phi)$:
    $$g_{ij} = \begin{pmatrix} 1 & 0 & 0 \\ 0 & r^2 & 0 \\ 0 & 0 & r^2\sin^2\theta \end{pmatrix}$$
  \end{ejemplo}
\end{frame}

\section{Resumen}

\begin{frame}{Resumen}
  \begin{block}{Conceptos Clave}
    \begin{itemize}
      \item \textbf{Tensor}: generalización multidimensional de escalares, vectores y matrices
      \item \textbf{Tipo $(p,q)$}: $p$ índices contravariantes, $q$ covariantes
      \item \textbf{Producto tensorial}: construye espacios de tensores
      \item \textbf{Contracción}: reduce el rango del tensor
      \item \textbf{Transformación}: reglas específicas bajo cambio de coordenadas
    \end{itemize}
  \end{block}
  
  \begin{alertblock}{Importancia}
    Los tensores son fundamentales en física moderna, geometría diferencial, y aplicaciones en ingeniería y ciencia de datos.
  \end{alertblock}
\end{frame}

\begin{frame}{Referencias}
  \begin{thebibliography}{99}
    \bibitem{kolecki} J. Kolecki, \textit{An Introduction to Tensors for Students of Physics and Engineering}, NASA Technical Memorandum, 2002.
    
    \bibitem{wald} R. Wald, \textit{General Relativity}, University of Chicago Press, 1984.
    
    \bibitem{lee} J. Lee, \textit{Introduction to Smooth Manifolds}, 2nd ed., Springer, 2012.
    
    \bibitem{schutz} B. Schutz, \textit{Geometrical Methods of Mathematical Physics}, Cambridge University Press, 1980.
  \end{thebibliography}
\end{frame}

\begin{frame}
  \centering
  \Huge ¿Preguntas?
  
  \vspace{1cm}
  
  \Large Gracias por su atención
\end{frame}

\end{document}
