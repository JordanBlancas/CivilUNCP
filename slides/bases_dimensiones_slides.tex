\documentclass[aspectratio=169]{beamer}
% slides/_preamble_slides.tex
% Tema moderno minimalista
\usetheme{default}
\usecolortheme{default}

\usepackage[spanish]{babel}
\usepackage[utf8]{inputenc}
\usepackage[T1]{fontenc}
\usepackage{lmodern}
\usepackage{graphicx}
\usepackage{ifthen}
\graphicspath{{figuras/}{./figuras/}{./}}

% Colores modernos
\definecolor{primaryColor}{RGB}{0, 102, 204}      % Azul profesional
\definecolor{secondaryColor}{RGB}{255, 87, 51}    % Coral/naranja moderno
\definecolor{accentColor}{RGB}{46, 204, 113}      % Verde menta
\definecolor{darkGray}{RGB}{44, 62, 80}           % Gris oscuro
\definecolor{lightGray}{RGB}{236, 240, 241}       % Gris claro
\definecolor{textColor}{RGB}{33, 37, 41}          % Negro suave

% Configuración del tema
\setbeamercolor{structure}{fg=primaryColor}
\setbeamercolor{palette primary}{bg=primaryColor,fg=white}
\setbeamercolor{palette secondary}{bg=secondaryColor,fg=white}
\setbeamercolor{palette tertiary}{bg=darkGray,fg=white}
\setbeamercolor{palette quaternary}{bg=accentColor,fg=white}

% Título y frames
\setbeamercolor{titlelike}{parent=palette primary}
\setbeamercolor{frametitle}{parent=palette primary}
\setbeamercolor{block title}{bg=primaryColor,fg=white}
\setbeamercolor{block body}{bg=lightGray,fg=textColor}
\setbeamercolor{block title alerted}{bg=secondaryColor,fg=white}
\setbeamercolor{block body alerted}{bg=lightGray,fg=textColor}
\setbeamercolor{block title example}{bg=accentColor,fg=white}
\setbeamercolor{block body example}{bg=lightGray,fg=textColor}

% Items y enumeración
\setbeamercolor{item}{fg=primaryColor}
\setbeamercolor{subitem}{fg=secondaryColor}
\setbeamercolor{subsubitem}{fg=accentColor}

% Estilo de los frames
\setbeamertemplate{frametitle}{
  \vspace{0.5em}
  \insertframetitle
  \vspace{0.3em}
}

% Bloques con esquinas redondeadas
\setbeamertemplate{blocks}[rounded][shadow=false]

% Items modernos
\setbeamertemplate{itemize items}[circle]
\setbeamertemplate{enumerate items}[circle]

% Sin símbolos de navegación
\setbeamertemplate{navigation symbols}{}

% Footline moderno - color único azul oscuro
\setbeamertemplate{footline}{
  \leavevmode%
  \hbox{%
    \begin{beamercolorbox}[wd=\paperwidth,ht=2.5ex,dp=1ex,leftskip=.3cm,rightskip=.3cm]{palette tertiary}%
      \usebeamerfont{author in head/foot}\insertshortauthor
      \hfill
      \usebeamerfont{title in head/foot}\insertshorttitle
      \hfill
      \usebeamerfont{date in head/foot}\insertshortdate{}
      \hspace*{2em}
      \insertframenumber{} / \inserttotalframenumber
    \end{beamercolorbox}}%
  \vskip0pt%
}

% Título con diseño moderno
\setbeamertemplate{title page}{
  \vbox{}
  \vfill
  \begingroup
    \centering
    \begin{beamercolorbox}[sep=8pt,center]{title}
      \usebeamerfont{title}\inserttitle\par%
      \ifx\insertsubtitle\@empty%
      \else%
        \vskip0.25em%
        {\usebeamerfont{subtitle}\usebeamercolor[fg]{subtitle}\insertsubtitle\par}%
      \fi%     
    \end{beamercolorbox}%
    \vskip1em\par
    \begin{beamercolorbox}[sep=8pt,center]{author}
      \usebeamerfont{author}\insertauthor
    \end{beamercolorbox}
    \begin{beamercolorbox}[sep=8pt,center]{institute}
      \usebeamerfont{institute}\insertinstitute
    \end{beamercolorbox}
    \begin{beamercolorbox}[sep=8pt,center]{date}
      \usebeamerfont{date}\insertdate
    \end{beamercolorbox}\vskip0.5em
  \endgroup
  \vfill
}

% Tipografía moderna
\setbeamerfont{title}{size=\Large,series=\bfseries}
\setbeamerfont{subtitle}{size=\large}
\setbeamerfont{frametitle}{size=\large,series=\bfseries}
\setbeamerfont{block title}{size=\normalsize,series=\bfseries}

\hypersetup{colorlinks=true, linkcolor=primaryColor, urlcolor=primaryColor, citecolor=secondaryColor}

% Definir entornos en español
\newtheorem{teorema}{Teorema}
\newtheorem{corolario}{Corolario}
\theoremstyle{definition}
\newtheorem{ejemplo}{Ejemplo}
\newtheorem{ejemplos}{Ejemplos}

\usepackage{amsmath, amssymb, amsthm}
\usepackage{listings}

\lstset{
  language=Python,
  basicstyle=\ttfamily\scriptsize,
  keywordstyle=\color{blue},
  commentstyle=\color{green!60!black},
  stringstyle=\color{red},
  showstringspaces=false,
  breaklines=true,
  frame=single,
  numbers=left,
  numberstyle=\tiny\color{gray},
  backgroundcolor=\color{lightGray}
}

\title{Bases, Dimensiones y Espacios Generados}
\subtitle{Álgebra Lineal}
\author{Jordan Blancas}
\date{\today}

\begin{document}

\begin{frame}
  \titlepage
\end{frame}

\section{Motivación}

\begin{frame}{¿Por qué estudiar bases y dimensiones?}
  \begin{block}{Problema práctico}
    En una estructura:
    \begin{itemize}
      \item ¿Cuántos movimientos \textit{independientes} puede hacer?
      \item ¿Cómo describir cualquier desplazamiento usando movimientos básicos?
    \end{itemize}
  \end{block}
  
  \begin{alertblock}{La respuesta}
    \begin{itemize}
      \item \textbf{Base}: conjunto de movimientos básicos independientes
      \item \textbf{Dimensión}: número mínimo de movimientos necesarios
    \end{itemize}
  \end{alertblock}
\end{frame}

\section{Combinaciones y Espacios Generados}

\begin{frame}{Combinaciones Lineales}
  \begin{block}{Definición}
    Una \textbf{combinación lineal} de vectores $v_1, \ldots, v_k$ es:
    $$w = \alpha_1 v_1 + \alpha_2 v_2 + \cdots + \alpha_k v_k$$
  \end{block}
  
  \begin{ejemplo}
    En $\mathbb{R}^2$: $(5, 3) = 5(1, 0) + 3(0, 1)$
  \end{ejemplo}
  
  \begin{ejemplos}
    \begin{itemize}
      \item En $\mathbb{R}^3$: $(6, 7, 11) = 2(1, 2, 3) + 4(1, 1, 2)$
      \item Puedes crear infinitos vectores "mezclando" los vectores base
    \end{itemize}
  \end{ejemplos}
\end{frame}

\begin{frame}{Espacio Generado}
  \begin{block}{Pregunta clave}
    Dado un conjunto de vectores $\{v_1, \ldots, v_k\}$, ¿qué vectores puedo alcanzar usando combinaciones lineales?
  \end{block}
  
  \begin{block}{Definición}
    El \textbf{espacio generado} (o span) es:
    $$\text{Gen}(v_1, \ldots, v_k) = \{\alpha_1 v_1 + \cdots + \alpha_k v_k : \alpha_i \in \mathbb{R}\}$$
    
    Es el conjunto de TODAS las combinaciones lineales posibles.
  \end{block}
  
  \begin{alertblock}{Propiedad importante}
    $\text{Gen}(v_1, \ldots, v_k)$ siempre es un subespacio vectorial.
  \end{alertblock}
\end{frame}

\begin{frame}{Ejemplos de Espacios Generados}
  \begin{ejemplos}
    \begin{itemize}
      \item $\text{Gen}((1, 0), (0, 1)) = \mathbb{R}^2$ (todo el plano)
      \item $\text{Gen}((1, 2)) = \{(t, 2t) : t \in \mathbb{R}\}$ (recta $y = 2x$)
      \item $\text{Gen}((1, 0, 0), (0, 1, 0)) = $ plano $xy$ en $\mathbb{R}^3$
      \item $\text{Gen}((1, 1, 1)) = \{(t, t, t) : t \in \mathbb{R}\}$ (recta diagonal)
    \end{itemize}
  \end{ejemplos}
  
  \begin{block}{Interpretación geométrica}
    \begin{itemize}
      \item 1 vector → genera una recta
      \item 2 vectores LI → generan un plano
      \item 3 vectores LI en $\mathbb{R}^3$ → generan todo el espacio
    \end{itemize}
  \end{block}
\end{frame}

\begin{frame}{Preguntas sobre Espacios Generados}
  \begin{block}{¿Está un vector en el espacio generado?}
    Pregunta: ¿$(7, 5)$ está en $\text{Gen}((1, 2), (2, 1))$?
    
    Equivale a: ¿existen $\alpha, \beta$ tales que:
    $$(7, 5) = \alpha(1, 2) + \beta(2, 1)$$
    
    Sistema: $\alpha + 2\beta = 7$ y $2\alpha + \beta = 5$
    
    Solución: $\alpha = 1, \beta = 3$ → SÍ está
  \end{block}
  
  \begin{block}{¿Cuántos vectores necesito?}
    Si añades vectores redundantes (LD), el espacio generado NO crece.
    
    $\text{Gen}((1,0), (0,1), (2,0)) = \text{Gen}((1,0), (0,1)) = \mathbb{R}^2$
  \end{block}
\end{frame}

\begin{frame}{Independencia Lineal}
  \begin{block}{Idea}
    Vectores son \textbf{linealmente independientes (LI)} si ninguno es combinación de los demás (no hay redundancia).
  \end{block}
  
  \begin{block}{Definición formal}
    $\{v_1, \ldots, v_k\}$ son LI si:
    $$\alpha_1 v_1 + \cdots + \alpha_k v_k = 0 \quad \Rightarrow \quad \alpha_1 = \cdots = \alpha_k = 0$$
  \end{block}
  
  \begin{ejemplos}
    \begin{itemize}
      \item $(1, 0)$ y $(0, 1)$ son LI $\checkmark$
      \item $(1, 0)$ y $(2, 0)$ son LD (misma dirección) $\times$
      \item $(1, 2)$ y $(2, 4)$ son LD (uno es múltiplo del otro) $\times$
    \end{itemize}
  \end{ejemplos}
\end{frame}

\section{Base}

\begin{frame}{Base: Definición}
  \begin{block}{Definición}
    Una \textbf{base} $\mathcal{B} = \{v_1, \ldots, v_n\}$ de $V$ cumple:
    \begin{enumerate}
      \item $\mathcal{B}$ es linealmente independiente
      \item $\text{Gen}(\mathcal{B}) = V$ (genera todo el espacio)
    \end{enumerate}
  \end{block}
  
  \begin{alertblock}{Conjunto mínimo e independiente que genera todo}
  \end{alertblock}
  
  \begin{ejemplos}
    \begin{itemize}
      \item $\mathbb{R}^2$: $\{(1,0), (0,1)\}$ (base canónica)
      \item $\mathbb{R}^2$: $\{(1,1), (1,-1)\}$ (otra base válida)
      \item $\mathbb{R}^3$: $\{(1,0,0), (0,1,0), (0,0,1)\}$
      \item Polinomios grado $\leq 2$: $\{1, x, x^2\}$
    \end{itemize}
  \end{ejemplos}
\end{frame}

\begin{frame}{Propiedad de Unicidad}
  \begin{teorema}
    Si $\mathcal{B} = \{v_1, \ldots, v_n\}$ es base de $V$, entonces todo $v \in V$ se escribe de forma \textbf{ÚNICA}:
    $$v = c_1 v_1 + c_2 v_2 + \cdots + c_n v_n$$
  \end{teorema}
  
  \begin{block}{Coordenadas}
    Los números $(c_1, \ldots, c_n)$ son las \textbf{coordenadas} de $v$ en la base $\mathcal{B}$.
  \end{block}
  
  \begin{ejemplo}
    En $\mathbb{R}^2$ con base $\{(1,1), (1,-1)\}$:
    
    $(5, 3) = 4(1,1) + 1(1,-1)$
    
    Coordenadas: $(4, 1)$ en esta base
  \end{ejemplo}
\end{frame}

\section{Dimensión}

\begin{frame}{Dimensión}
  \begin{teorema}[Fundamental]
    Todas las bases de un mismo espacio tienen el mismo número de vectores.
  \end{teorema}
  
  \begin{block}{Definición}
    La \textbf{dimensión} $\dim(V)$ es el número de vectores en cualquier base de $V$.
  \end{block}
  
  \begin{ejemplos}
    \begin{itemize}
      \item $\dim(\mathbb{R}^2) = 2$, $\dim(\mathbb{R}^3) = 3$, $\dim(\mathbb{R}^n) = n$
      \item $\dim(P_2(\mathbb{R})) = 3$ (base: $\{1, x, x^2\}$)
      \item $\dim(M_{2 \times 3}) = 6$ (matrices $2 \times 3$)
    \end{itemize}
  \end{ejemplos}
  
  \begin{alertblock}{Interpretación física}
    Dimensión = número de grados de libertad independientes
  \end{alertblock}
\end{frame}

\begin{frame}{Reglas sobre Dimensión}
  \begin{block}{Si $\dim(V) = n$:}
    \begin{enumerate}
      \item Cualquier $n$ vectores LI forman una base
      \item Cualquier $n$ vectores que generan $V$ forman una base
      \item No puedes tener más de $n$ vectores LI en $V$
      \item Necesitas al menos $n$ vectores para generar $V$
    \end{enumerate}
  \end{block}
  
  \begin{ejemplo}
    En $\mathbb{R}^3$ ($\dim = 3$):
    \begin{itemize}
      \item 3 vectores LI → es base automáticamente
      \item 4 vectores → al menos uno es dependiente
      \item 2 vectores LI → no generan todo $\mathbb{R}^3$
    \end{itemize}
  \end{ejemplo}
\end{frame}

\section{Subespacios}

\begin{frame}{Subespacios}
  \begin{block}{Definición}
    $W \subseteq V$ es \textbf{subespacio} si:
    \begin{enumerate}
      \item $\vec{0} \in W$
      \item Cerrado bajo suma y multiplicación escalar
    \end{enumerate}
  \end{block}
  
  \begin{ejemplos}
    En $\mathbb{R}^3$:
    \begin{itemize}
      \item Recta por el origen: $\dim = 1$
      \item Plano por el origen: $\dim = 2$
      \item Todo $\mathbb{R}^3$: $\dim = 3$
    \end{itemize}
  \end{ejemplos}
  
  \begin{alertblock}{Clave}
    $\dim(W) \leq \dim(V)$ si $W \subseteq V$
  \end{alertblock}
\end{frame}

\section{Código}

\begin{frame}[fragile]{Código Python Básico}
\begin{lstlisting}
import numpy as np

# Vectores en R^3
v1 = np.array([1, 2, 3])
v2 = np.array([4, 5, 6])
v3 = np.array([7, 8, 9])

# Formar matriz (vectores como columnas)
A = np.column_stack([v1, v2, v3])

# Calcular rango
rango = np.linalg.matrix_rank(A)
num_vectores = A.shape[1]

print(f"Rango: {rango}")
print(f"Numero de vectores: {num_vectores}")

if rango == num_vectores:
    print("Los vectores son LI (forman base)")
else:
    print("Los vectores son LD (dependientes)")
    
# La dimension es el rango de la matriz
print(f"Dimension del espacio generado: {rango}")
\end{lstlisting}
\end{frame}

\section{Resumen}

\begin{frame}{Resumen}
  \begin{enumerate}
    \item \textbf{Combinación lineal}: $\alpha_1 v_1 + \cdots + \alpha_k v_k$
    \item \textbf{Espacio generado}: todas las combinaciones lineales de un conjunto
    \item \textbf{Independencia lineal}: vectores sin redundancia
    \item \textbf{Base}: conjunto mínimo LI que genera todo
    \item \textbf{Dimensión}: número de vectores en cualquier base
    \item \textbf{Subespacios}: espacios más pequeños dentro de $V$
  \end{enumerate}
  
  \begin{alertblock}{Clave}
    Base = conjunto LI que genera todo $\quad \Rightarrow \quad$ dimensión = tamaño de la base
  \end{alertblock}
\end{frame}

\end{document}
